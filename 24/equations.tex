\documentclass{article}
\begin{document}


Trajectories intersect when $X_a = X_b$ and $Y_a = Y_b$

Positions for each hailstone can be calculated:

\begin{displaymath}
    X = X_0 + V_x \times t
\end{displaymath}
\begin{displaymath}
    Y = Y_0 + V_y \times t
\end{displaymath}

Then:
\begin{equation}
    X_{a0} + V_{ax} \times t_a = X_{b0} + V_{bx} \times t_b
\end{equation}
\begin{equation}
    Y_{a0} + V_{ay} \times t_a = Y_{b0} + V_{by} \times t_b
\end{equation}

Rearange $V \times t$
\begin{displaymath}
    V_{ax} \times t_a = X_{b0} + V_{bx} \times t_b - X_{a0}
\end{displaymath}
\begin{displaymath}
    V_{ay} \times t_a = Y_{b0} + V_{by} \times t_b - Y_{a0}
\end{displaymath}

Extract $t_a$
\begin{displaymath}
    t_a = \frac{X_{b0} + V_{bx} \times t_b - X_{a0}}{V_{ax}}
\end{displaymath}

\begin{displaymath}
    t_a = \frac{Y_{b0} + V_{by} \times t_b - Y_{a0}}{V_{ay}}
\end{displaymath}

Both sides
\begin{displaymath}
    \frac{X_{b0} + V_{bx} \times t_b - X_{a0}}{V_{ax}} = \frac{Y_{b0} + V_{by} \times t_b - Y_{a0}}{V_{ay}}
\end{displaymath}

Rearange fractions
\begin{displaymath}
    \frac{X_{b0} - X_{a0}}{V_{ax}} + \frac{V_{bx} \times t_b}{V_{ax}} = \frac{Y_{b0} - Y_{a0}}{V_{ay}} + \frac{V_{by} \times t_b}{V_{ay}}
\end{displaymath}
    

Move consts to right, vars to left
\begin{displaymath}
    \frac{V_{bx} \times t_b}{V_{ax}} - \frac{V_{by} \times t_b}{V_{ay}} = \frac{Y_{b0} - Y_{a0}}{V_{ay}} - \frac{X_{b0} - X_{a0}}{V_{ax}}
\end{displaymath}
    

Extact $t_b$
\begin{displaymath}
    t_b \times (\frac{V_{bx}}{V_{ax}} - \frac{V_{by}}{V_{ay}}) = \frac{Y_{b0} - Y_{a0}}{V_{ay}} - \frac{X_{b0} - X_{a0}}{V_{ax}}
\end{displaymath}


Re-arrange subtractions:
\begin{displaymath}
    t_b \times (\frac{V_{bx} \times V_{ay} - V_{by} \times V_{ax}}{V_{ax} \times V_{ay}}) = \frac{(Y_{b0} - Y_{a0}) \times V_{ax} - (X_{b0} - X_{a0}) \times V_{ay}}{V_{ax} \times V_{ay}}
\end{displaymath}

Extact $t_b$
\begin{displaymath}
    t_b = \frac{\frac{(Y_{b0} - Y_{a0}) \times V_{ax} - (X_{b0} - X_{a0}) \times V_{ay}}{V_{ax} \times V_{ay}}}{\frac{V_{bx} \times V_{ay} - V_{by} \times V_{ax}}{V_{ax} \times V_{ay}}}
\end{displaymath}

Simplify
\begin{equation}
    t_b = \frac{(Y_{b0} - Y_{a0}) \times V_{ax} - (X_{b0} - X_{a0}) \times V_{ay}}{V_{bx} \times V_{ay} - V_{by} \times V_{ax}}
\end{equation}

Rearranging (1) and (2) for b:
\begin{displaymath}
    V_{bx} \times t_b = X_{a0} + V_{ax} \times t_a - X_{b0}
\end{displaymath}
\begin{displaymath}
    V_{by} \times t_b = Y_{a0} + V_{ay} \times t_a - Y_{b0}
\end{displaymath}

Which is symmetric, so the expression for $t_a$ must be
\begin{displaymath}
    t_a = \frac{(Y_{a0} - Y_{b0}) \times V_{bx} - (X_{a0} - X_{b0}) \times V_{by}}{V_{ax} \times V_{by} - V_{ay} \times V_{bx}}
\end{displaymath}

Programmatically it would probably be easier to just swap a and b. Anyway, we're only interested in situations when $t_a > 0$ and $t_b > 0$.
Additionally we're interested ony on $V_{bx} \times V_{ay} \neq V_{by} \times V_{ax}$, because otherwise these are parallel lines.

To determine if there will be an intersection at some point, we should calculate both divisor and divident and compare their signs.

Additional requirement was to determine if they will intersect inside some range of coordinates, so we have to actually calculate the
point of intersection. As there are large numbers and I'd like to avoid floating-point imprecisions as much as possible, let's try to plug the
formula for $t_b$ into the coordinate calculations and see if it could be simplified.



\begin{displaymath}
    X_b = X_{0b} + V_{bx} \times t_b
\end{displaymath}


\begin{displaymath}
    X_b = X_{0b} + V_{bx} \times \frac{(Y_{b0} - Y_{a0}) \times V_{ax} - (X_{b0} - X_{a0}) \times V_{ay}}{V_{bx} \times V_{ay} - V_{by} \times V_{ax}}
\end{displaymath}


Ok, it does not look like it can be simplified much, let's only multiply the divident with $V_{bx}$ first.

\begin{displaymath}
    X_b = X_{0b} + \frac{V_{bx} \times ((Y_{b0} - Y_{a0}) \times V_{ax} - (X_{b0} - X_{a0}) \times V_{ay})}{V_{bx} \times V_{ay} - V_{by} \times V_{ax}}
\end{displaymath}







\end{document}

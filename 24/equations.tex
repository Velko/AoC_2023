\documentclass{article}
\usepackage{amsmath}

\makeatletter
\renewcommand*\env@matrix[1][*\c@MaxMatrixCols c]{%
  \hskip -\arraycolsep
  \let\@ifnextchar\new@ifnextchar
  \array{#1}}
\makeatother

\begin{document}

\section{Part 1 - XY trajectory intersection}

Trajectories intersect when $X_a = X_b$ and $Y_a = Y_b$

Positions for each hailstone can be calculated:

\begin{displaymath}
    X = X_0 + V_x \times t
\end{displaymath}
\begin{displaymath}
    Y = Y_0 + V_y \times t
\end{displaymath}


This can be expressed by system of equations:
\begin{equation} \label{eq:1}
    \begin{cases}
    X_{a0} + V_{ax} \times t_a = X_{b0} + V_{bx} \times t_b \\
    Y_{a0} + V_{ay} \times t_a = Y_{b0} + V_{by} \times t_b
    \end{cases}
\end{equation}

\subsection{Solve by Elimination of variables}

Rearange $V \times t$
\begin{displaymath}
    \begin{cases}
    V_{ax} \times t_a = X_{b0} + V_{bx} \times t_b - X_{a0} \\
    V_{ay} \times t_a = Y_{b0} + V_{by} \times t_b - Y_{a0}
    \end{cases}
\end{displaymath}

Extract $t_a$
\begin{displaymath}
    \begin{cases}
    t_a = \frac{X_{b0} + V_{bx} \times t_b - X_{a0}}{V_{ax}} \\
    t_a = \frac{Y_{b0} + V_{by} \times t_b - Y_{a0}}{V_{ay}}
    \end{cases}
\end{displaymath}

Both sides
\begin{displaymath}
    \frac{X_{b0} + V_{bx} \times t_b - X_{a0}}{V_{ax}} = \frac{Y_{b0} + V_{by} \times t_b - Y_{a0}}{V_{ay}}
\end{displaymath}

Rearange fractions
\begin{displaymath}
    \frac{X_{b0} - X_{a0}}{V_{ax}} + \frac{V_{bx} \times t_b}{V_{ax}} = \frac{Y_{b0} - Y_{a0}}{V_{ay}} + \frac{V_{by} \times t_b}{V_{ay}}
\end{displaymath}
    

Move consts to right, vars to left
\begin{displaymath}
    \frac{V_{bx} \times t_b}{V_{ax}} - \frac{V_{by} \times t_b}{V_{ay}} = \frac{Y_{b0} - Y_{a0}}{V_{ay}} - \frac{X_{b0} - X_{a0}}{V_{ax}}
\end{displaymath}
    

Extact $t_b$
\begin{displaymath}
    t_b \times (\frac{V_{bx}}{V_{ax}} - \frac{V_{by}}{V_{ay}}) = \frac{Y_{b0} - Y_{a0}}{V_{ay}} - \frac{X_{b0} - X_{a0}}{V_{ax}}
\end{displaymath}


Re-arrange subtractions:
\begin{displaymath}
    t_b \times (\frac{V_{bx} \times V_{ay} - V_{by} \times V_{ax}}{V_{ax} \times V_{ay}}) = \frac{(Y_{b0} - Y_{a0}) \times V_{ax} - (X_{b0} - X_{a0}) \times V_{ay}}{V_{ax} \times V_{ay}}
\end{displaymath}

Extact $t_b$
\begin{displaymath}
    t_b = \frac{\frac{(Y_{b0} - Y_{a0}) \times V_{ax} - (X_{b0} - X_{a0}) \times V_{ay}}{V_{ax} \times V_{ay}}}{\frac{V_{bx} \times V_{ay} - V_{by} \times V_{ax}}{V_{ax} \times V_{ay}}}
\end{displaymath}

Simplify
\begin{equation} \label{eq:2}
    t_b = \frac{(Y_{b0} - Y_{a0}) \times V_{ax} - (X_{b0} - X_{a0}) \times V_{ay}}{V_{bx} \times V_{ay} - V_{by} \times V_{ax}}
\end{equation}

Rearranging (\ref{eq:1}) for b:
\begin{displaymath}
    \begin{cases}
    V_{bx} \times t_b = X_{a0} + V_{ax} \times t_a - X_{b0} \\
    V_{by} \times t_b = Y_{a0} + V_{ay} \times t_a - Y_{b0}
    \end{cases}
\end{displaymath}

Which is symmetric, so the expression for $t_a$ must be similar to (\ref{eq:2}), with a and b swapped:
\begin{displaymath}
    t_a = \frac{(Y_{a0} - Y_{b0}) \times V_{bx} - (X_{a0} - X_{b0}) \times V_{by}}{V_{ax} \times V_{by} - V_{ay} \times V_{bx}}
\end{displaymath}

Programmatically it would probably be easier to just swap the input variables. Anyway, we're only interested in situations when $t_a > 0$ and $t_b > 0$.
Additionally we're interested ony on $V_{bx} \times V_{ay} \neq V_{by} \times V_{ax}$, because otherwise these are parallel lines.

To determine if there will be an intersection at some point, we should calculate both divisor and divident and compare their signs.

Additional requirement was to determine if they will intersect inside some range of coordinates, so we have to actually calculate the
point of intersection. As there are large numbers and I'd like to avoid floating-point imprecisions as much as possible, let's try to plug the
formula for $t_b$ into the coordinate calculations and see if it could be simplified.



\begin{displaymath}
    X_b = X_{0b} + V_{bx} \times t_b
\end{displaymath}


\begin{displaymath}
    X_b = X_{0b} + V_{bx} \times \frac{(Y_{b0} - Y_{a0}) \times V_{ax} - (X_{b0} - X_{a0}) \times V_{ay}}{V_{bx} \times V_{ay} - V_{by} \times V_{ax}}
\end{displaymath}


Ok, it does not look like it can be simplified much, let's only multiply the dividend with $V_{bx}$ first.

\begin{displaymath}
    X_b = X_{0b} + \frac{V_{bx} \times ((Y_{b0} - Y_{a0}) \times V_{ax} - (X_{b0} - X_{a0}) \times V_{ay})}{V_{bx} \times V_{ay} - V_{by} \times V_{ax}}
\end{displaymath}

On the other hand, multiplying dividend with $V_{bx}$ may lead to integer overflow. One might try to risk with floating point imprecisions instead.

The perfect solution calls for math utilizing longer than 64-bit integers and some additional logic if intersection happens to be exactly on
the border.


\subsection{Solve by Row reduction}

The solution above works, but getting to that involves a lot of manual formula conversions. It worked but I'd like to practice a little
by solving it in a more computer-oriented way. I'm considering it as a practice for \ref{part-2}.

Take (\ref{eq:1}) again and re-arrange it to a typical layout of \emph{System of Linear equations}:

\begin{displaymath}
    \begin{cases}
        V_{ax} \times t_a - V_{bx} \times t_b = X_{b0} - X_{a0} \\
        V_{ay} \times t_a - V_{by} \times t_b = Y_{b0} - Y_{a0}
    \end{cases}
\end{displaymath}

This can be rewritten in \emph{Augmented matrix} form:

\begin{displaymath}
\begin{bmatrix}[cc|c]
    V_{ax} & - V_{bx} & X_{b0} - X_{a0} \\
    V_{ay} & - V_{by} & Y_{b0} - Y_{a0}
\end{bmatrix}
\end{displaymath}

With all coefficients plugged in, this should be calculated numerically by a program.

Let's practice on sample data, to understand the algorithm better:

\begin{itemize}
\item 19, 13, 30 @ -2, 1, -2
\item 18, 19, 22 @ -1, -1, -2
\end{itemize}

\begin{displaymath}
    \begin{bmatrix}[cc|c]
        -2 & - -1 & 18 - 19 \\
         1 & - -1 & 19 - 13
    \end{bmatrix}
\end{displaymath}


\begin{displaymath}
    \begin{bmatrix}[cc|c]
        -2 & 1 & -1 \\
         1 & 1 &  6
    \end{bmatrix}
\end{displaymath}

Divide 1-st row by -2:

\begin{displaymath}
    \begin{bmatrix}[cc|c]
         1 & -\frac{1}{2} & \frac{1}{2} \\
         1 & 1 &  6
    \end{bmatrix}
\end{displaymath}

Subtract 1-st row (multiplied by 1) from the 2nd row:


\begin{displaymath}
    \begin{bmatrix}[cc|c]
         1 & -\frac{1}{2} & \frac{1}{2} \\
         0 & 1\frac{1}{2} &  5\frac{1}{2}
    \end{bmatrix}
\end{displaymath}


Divide 2nd row by 1$\frac{1}{2}$

\begin{displaymath}
    \begin{bmatrix}[cc|c]
         1 & -\frac{1}{2} & \frac{1}{2} \\
         0 & 1 &  \frac{11}{3}
    \end{bmatrix}
\end{displaymath}


Subtract 2nd row multiplied by -$\frac{1}{2}$ from the first:

\begin{displaymath}
    \begin{bmatrix}[cc|c]
         1 & 0 & \frac{7}{3} \\
         0 & 1 &  \frac{11}{3}
    \end{bmatrix}
\end{displaymath}

Here we get that $t_a = \frac{7}{3}$ and $t_b = \frac{11}{3}$.

Plugging $t_a$ into coordinate calculations for \emph{A}:

\begin{displaymath}
    \begin{cases}
    X = 19 + (-2 * \frac{7}{3}) = 14.333 \\
    Y = 13 + (1 * \frac{7}{3}) = 15.333
    \end{cases}
\end{displaymath}


Do the same with $t_b$:

\begin{displaymath}
    \begin{cases}
    X = 18 + (-1 * \frac{11}{3}) = 14.333 \\
    Y = 19 + (-1 * \frac{11}{3}) = 15.333
    \end{cases}
\end{displaymath}

This proves that the algorithm works, it only needs to be coded.



\section{Part 2 - Impact trajectory in 3D} \label{part-2}

It is known that there is a straight line trajectory that collides with all of the hailstones.

The coordinate calculations for straight line trajectories remains same as in Part 1, just we have to also consider Z coordinate.

\begin{displaymath}
    \begin{cases}
        X = X_{0} + V_{x} \\
        Y = Y_{0} + V_{y} \\
        Z = Z_{0} + V_{z}
    \end{cases}
\end{displaymath}

At impact with hailstone A, the coordinates of the rock ($X_r$, $Y_r$ and $Z_r$) should be equal to the coordinates of hailstone ($X_a$, $Y_a$ and $Z_a$)
at same moment of time $t_a$.

\begin{equation}
    \begin{cases}
        X_{r0} + V_{rx} \times t_a = X_{a0} + V_{ax} \times t_a \\
        Y_{r0} + V_{ry} \times t_a = Y_{a0} + V_{ay} \times t_a \\
        Z_{r0} + V_{rz} \times t_a = Z_{a0} + V_{az} \times t_a \\
    \end{cases}
\end{equation}

Move $t_a$ to the left side, components not dependent on it - to the right.
\begin{displaymath}
    \begin{cases}
        V_{rx} \times t_a - V_{ax} \times t_a = X_{a0} - X_{r0} \\
        V_{ry} \times t_a - V_{ay} \times t_a = Y_{a0} - Y_{r0} \\
        V_{rz} \times t_a - V_{az} \times t_a = Z_{a0} - Z_{r0} \\
    \end{cases}
\end{displaymath}

Express $t_a$:

\begin{displaymath}
    \begin{cases}
        t_a = \frac{X_{a0} - X_{r0}}{V_{rx} - V_{ax}} \\
        t_a = \frac{Y_{a0} - Y_{r0}}{V_{ry} - V_{ay}} \\
        t_a = \frac{Z_{a0} - Z_{r0}}{V_{rz} - V_{az}} \\
    \end{cases}
\end{displaymath}

Let's focus on X and Y coordinates for now.
\begin{displaymath}
    \frac{X_{a0} - X_{r0}}{V_{rx} - V_{ax}} = \frac{Y_{a0} - Y_{r0}}{V_{ry} - V_{ay}}\\
\end{displaymath}


Get rid of fractions:
\begin{displaymath}
    (X_{a0} - X_{r0}) \times (V_{ry} - V_{ay}) = (Y_{a0} - Y_{r0}) \times (V_{rx} - V_{ax})\\
\end{displaymath}

Get rid of parentheses:
\begin{displaymath}
    X_{a0} \times V_{ry} - X_{a0} \times V_{ay} - X_{r0} \times V_{ry} + X_{r0} \times V_{ay} = Y_{a0} \times V_{rx} - Y_{a0} \times V_{ax} - Y_{r0} \times V_{rx} + Y_{r0} \times V_{ax}\\
\end{displaymath}

Re-arrange so that seemingly non-linear part is on the left side.
\begin{displaymath}
    Y_{r0} \times V_{rx} - X_{r0} \times V_{ry} = Y_{a0} \times V_{rx} - Y_{a0} \times V_{ax} + Y_{r0} \times V_{ax} - X_{a0} \times V_{ry} + X_{a0} \times V_{ay} - X_{r0} \times V_{ay}\\
\end{displaymath}


The $Y_{r0} \times V_{rx} - X_{r0} \times V_{ry}$ part may seem non-linear, but it will be the same for each hailstone. Let's introduce
hailstone B:
\begin{displaymath}
    \begin{aligned}
    Y_{a0} \times V_{rx} - Y_{a0} \times V_{ax} + Y_{r0} \times V_{ax} - X_{a0} \times V_{ry} + X_{a0} \times V_{ay} - X_{r0} \times V_{ay}\\
  = Y_{b0} \times V_{rx} - Y_{b0} \times V_{bx} + Y_{r0} \times V_{bx} - X_{b0} \times V_{ry} + X_{b0} \times V_{by} - X_{r0} \times V_{by}\\
    \end{aligned}
\end{displaymath}

Move stuff around again: constants to the right, unknowns to the left.

\begin{displaymath}
    \begin{aligned}
        Y_{a0} \times V_{rx} + Y_{r0} \times V_{ax} - X_{a0} \times V_{ry} - X_{r0} \times V_{ay} - Y_{b0} \times V_{rx} - Y_{r0} \times V_{bx} \\
        + X_{b0} \times V_{ry} + X_{r0} \times V_{by} = - Y_{b0} \times V_{bx} + X_{b0} \times V_{by} + Y_{a0} \times V_{ax} - X_{a0} \times V_{ay}\\
    \end{aligned}
\end{displaymath}

Group them:

\begin{displaymath}
    \begin{aligned}
        (V_{by} - V_{ay}) \times X_{r0} + (V_{ax} - V_{bx}) \times Y_{r0} + (Y_{a0} - Y_{b0}) \times V_{rx} + (X_{b0} - X_{a0}) \times V_{ry} \\
        = X_{b0} \times V_{by} - Y_{b0} \times V_{bx} - X_{a0} \times V_{ay} + Y_{a0} \times V_{ax}\\
    \end{aligned}
\end{displaymath}

Now we have an equation with 4 unknowns. To solve it, we need 4 equations. Let's introduce additional hailstones C, D and E

\begin{displaymath}
    \begin{cases}
        (V_{by} - V_{ay}) \times X_{r0} + (V_{ax} - V_{bx}) \times Y_{r0} + (Y_{a0} - Y_{b0}) \times V_{rx} + (X_{b0} - X_{a0}) \times V_{ry} = X_{b0} \times V_{by} - Y_{b0} \times V_{bx} - X_{a0} \times V_{ay} + Y_{a0} \times V_{ax}\\
        (V_{cy} - V_{ay}) \times X_{r0} + (V_{ax} - V_{cx}) \times Y_{r0} + (Y_{a0} - Y_{c0}) \times V_{rx} + (X_{c0} - X_{a0}) \times V_{ry} = X_{c0} \times V_{cy} - Y_{c0} \times V_{cx} - X_{a0} \times V_{ay} + Y_{a0} \times V_{ax}\\
        (V_{dy} - V_{ay}) \times X_{r0} + (V_{ax} - V_{dx}) \times Y_{r0} + (Y_{a0} - Y_{d0}) \times V_{rx} + (X_{d0} - X_{a0}) \times V_{ry} = X_{d0} \times V_{dy} - Y_{d0} \times V_{dx} - X_{a0} \times V_{ay} + Y_{a0} \times V_{ax}\\
        (V_{ey} - V_{ay}) \times X_{r0} + (V_{ax} - V_{ex}) \times Y_{r0} + (Y_{a0} - Y_{e0}) \times V_{rx} + (X_{e0} - X_{a0}) \times V_{ry} = X_{e0} \times V_{ey} - Y_{e0} \times V_{ex} - X_{a0} \times V_{ay} + Y_{a0} \times V_{ax}\\
    \end{cases}
\end{displaymath}

Now there are 4 equations and 4 unknowns, this should be solvable. Let's compose it into a matrix:

\begin{displaymath}
    \begin{bmatrix}[cccc|c]
        V_{by} - V_{ay} & V_{ax} - V_{bx} & Y_{a0} - Y_{b0} & X_{b0} - X_{a0} & X_{b0} \times V_{by} - Y_{b0} \times V_{bx} - X_{a0} \times V_{ay} + Y_{a0} \times V_{ax}\\
        V_{cy} - V_{ay} & V_{ax} - V_{cx} & Y_{a0} - Y_{c0} & X_{c0} - X_{a0} & X_{c0} \times V_{cy} - Y_{c0} \times V_{cx} - X_{a0} \times V_{ay} + Y_{a0} \times V_{ax}\\
        V_{dy} - V_{ay} & V_{ax} - V_{dx} & Y_{a0} - Y_{d0} & X_{d0} - X_{a0} & X_{d0} \times V_{dy} - Y_{d0} \times V_{dx} - X_{a0} \times V_{ay} + Y_{a0} \times V_{ax}\\
        V_{ey} - V_{ay} & V_{ax} - V_{ex} & Y_{a0} - Y_{e0} & X_{e0} - X_{a0} & X_{e0} \times V_{ey} - Y_{e0} \times V_{ex} - X_{a0} \times V_{ay} + Y_{a0} \times V_{ax}\\
    \end{bmatrix}
\end{displaymath}

Gauss-elimination solver found the solution. This gives us position and velocity in XY plane. What about Z? Just repeat the process using XZ or YZ pairs of coordinates.

We're doing a lot of floating-point math on fairly big numbers, so there might be some precision loss. Let's calculate all possible combinations (XY, XZ and YZ)
and then use averages from them.




\end{document}
